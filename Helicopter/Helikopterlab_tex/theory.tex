\section{Theory}

% This file should include project description and a model of the helicopter
%%%%%%%%%%%%%%%%%%%%%%%%%%%%%%%%%%%%%%%%%%%%%%%%%%%%%%%%%%%%%%%%%%%%%%%%%%%%%%%%%%%%%%%%%%%%

%%%%                                PART 1 PROBLEM 1

%%%%%%%%%%%%%%%%%%%%%%%%%%%%%%%%%%%%%%%%%%%%%%%%%%%%%%%%%%%%%%%%%%%%%%%%%%%%%%%%%%%%%%%%%%%%
\subsection{Part 1: Feed Forward}
\subsubsection{Problem 1}

The relations between motor voltage, and actual force delivered are given by
\begin{subequations} \label{eq:part1_prob1_forces}
    \begin{align}
        F_f &= K_f V_f \\
        F_b &= K_f V_b
    \end{align}
\end{subequations}

where $F_b$ and $F_f$ are propeller forces, $V_b$ and $V_f$ are motor voltages and $K_f$ is the motor force constant. Our regulator will create two outputs, $V_s$ and $V_d$, which relate to the actual motor voltages like this:
\begin{subequations} \label{eq:voltages}
    \begin{align}
        V_s &= V_f + V_b \label{eq:voltages_vs}\\
        V_d &= V_f - V_b \label{eq:voltages_vd}
    \end{align}
\end{subequations}


We start with the momentum equations, derived from Newton's second law, 
\begin{equation} \label{eq:part1_prob1_momentum_general}
    \sum M = J \alpha
\end{equation}

where $M$ is torque, $J$ is  moment of inertia and $\alpha$ is angular acceleration. The moment around the pitch joint, $J_p\ddot{p}$, is affected by four forces. The propeller forces generated by their blades, and their weight. Since our propellers have the same weight, and are mounted on each side of the joint, the weight forces cancel each other out. 
Therefore only the forces generated by the blades affect the moment. Propellers are mounted a distance $l_p$ from the pitch joint, giving us the following expression for moment around the pitch joint:
\begin{equation}
    J_{p} \ddot{p} = l_p F_f - l_p F_p
\end{equation}

Substituting $F_f$ and $F_b$ with $V_d$ given from eqs. (\ref{eq:part1_prob1_forces}) and (\ref{eq:voltages_vd}) gives

\begin{equation}
    J_p \ddot{p} = l_p K_f V_d
\end{equation}

By defining $L_1 = l_p K_f$, we find the equation of motion for pitch to be

\begin{equation} \label{eq:model_pitch}
    J_p \ddot{p} = L_1 V_d
\end{equation}

The moment around the elevation joint is affected by the three gravitational forces from the propellers and the counterweight, and can be described by the equation
\begin{equation}
    J_e \ddot{e} = (F_{g, c} l_{c} - (F_{g, f} + F_{g, b}) l_h) \cos (e) + ((F_f + F_b) l_h) \cos(p)
\end{equation}

where $e$ is the elevation angle. Choosing $L_2 = F_{g, c} l_c - (F_{g, f} + F_{g, b}) l_h$, using eqs. (\ref{eq:part1_prob1_forces}) and (\ref{eq:voltages_vs}) to substitute $F_f$ and $F_b$ and then setting $L_3 = K_f l_h$, we get the second equation of motion on the desired form:

\begin{equation} \label{eq:model_elev}
    J_e \ddot{e} = L_2 \cos (e) + L_3 V_s \cos (p)
\end{equation}

%Momentum equation for elevation gives
%\begin{equation}
%    L_2 = F_{g, c} l_c - (F_{g, g} + F_{g, b}) l_h
%\end{equation}

%\begin{equation}
%    L_3 = K_f l_h
%\end{equation}

Only the propeller forces contribute to moment around the travel joint. The way the assignment defines direction for pitch and travel cause the need of a minus sign on the equation, given by
\begin{equation}
    J_{\lambda} \ddot{\lambda} = -(F_f + F_b) l_h l_p \sin (p) \cos (e)
\end{equation}

where $\lambda$ is the travel angle. Again, we substitute $F_f + F_b = K_f V_s$ and define $L_4 = -K_f l_h$ and the equation of motion becomes
\begin{equation}
    J_{\lambda} \ddot{\lambda} = L_4 V_s \sin (p) \cos (e)
\end{equation}

Below are all the equations that make up our model of the helicopter, with their constants:

\begin{subequations}\label{eq:model_al}
	\begin{align}
		J_p\ddot{p} &= L_{1}V_{d} \label{eq:model_se_al_pitch}\\
		J_e\ddot{e} &= L_{2} \cos(e) + L_3 V_s \cos(p) \label{eq:model_se_al_elev}\\
		J_\lambda \ddot{\lambda} &= L_4 V_s \cos(e) \sin(p) \label{eq:model_se_al_lambda}
	\end{align}
\end{subequations}

with 
\begin{subequations}
    \begin{align}
    L_1 &= K_f l_p \\
    L_2 &= F_{g, c} l_c - (F_{g, f} + F_{g, b}) l_h \\
    L_3 &= K_f l_h \\
    L_4 &= -K_f l_h
    \end{align}
\end{subequations}


%%%%%%%%%%%%%%%%%%%%%%%%%%%%%%%%%%%%%%%%%%%%%%%%%%%%%%%%%%%%%%%%%%%%%%%%%%%%%%%%%%%%%%%%%%%%

%%%%                                PART 1 PROBLEM 2

%%%%%%%%%%%%%%%%%%%%%%%%%%%%%%%%%%%%%%%%%%%%%%%%%%%%%%%%%%%%%%%%%%%%%%%%%%%%%%%%%%%%%%%%%%%%

\subsubsection{Problem 2}
As we can see from eq. (\ref{eq:model_al}), our system is not linear. The purpose of this lab is to control the system using linear control theory, which requires us to linearize the system. The system will be linearized around a equilibrium defined: $(p,\, e,\, \lambda)^\top = (p^\ast,\, e^\ast,\, \lambda ^\ast)^\top$ and $(V_d,\, V_s)^\top = (V_d^\ast,\, V_s^\ast)^\top$ with $p^\ast = e^\ast = \lambda ^\ast = 0$. Inserting $\ddot{p} = \ddot{\lambda} = \ddot{e} = 0$ in eq. (\ref{eq:model_al}) gives

\begin{subequations} \label{eq:voltage_equi}
    \begin{align}
        V_d^\ast &= 0 \label{eq:voltage_equi_1}\\
        V_s^\ast &= - \frac{L_2 \cos(e)}{L_3 \cos(p)}\label{eq:voltage_equi_2}
    \end{align}
\end{subequations}

We linearize using 
\begin{equation} \label{eq:linearization_general}
    \bm{\tilde{A}} = \frac{\partial \bm{f}}{\partial \bm{x}} \bigg\rvert_{\bm{x}^\ast, \, \bm{u}^\ast},\: \bm{\tilde{B}} = \frac{\partial \bm{f}}{\partial \bm{u}} \bigg\rvert_{\bm{x}^\ast, \, \bm{u}^\ast}
\end{equation}

where $\bm{f}$ represents the equations in eq. (\ref{eq:model_al}) divided by $J_p$, $J_e$ and $J_\lambda$, respectively \cite{Chen2014}.\medskip

Using eq. (\ref{eq:linearization_general}), we get the following matrices for our linearized system. 
\begin{equation}
    \setstackgap{L}{1.1\baselineskip}
    \fixTABwidth{T}
	\bm{\tilde A} = \bracketMatrixstack{
		0 &  1 &  0 &  0 &  0 &  0  \\
		0 &  0 &  0 &  0 &  0 &  0  \\
		0 &  0 &  0 &  1 &  0 &  0  \\
		0 &  0 &  0 &  0 &  0 &  0  \\
		0 &  0 &  0 &  0 &  0 &  1  \\
		-\frac{L_2 L_4}{L_3 J_\lambda} &  0 &  0 &  0 &  0 &  0 &                                
	}
\end{equation}
and

\begin{equation}
    \setstackgap{L}{1.1\baselineskip}
    \fixTABwidth{T}
	\bm{\tilde{B}} = \bracketMatrixstack{
		0               &  0 \\
		0               &  \frac{L_1}{J_p} \\
		0               &  0 \\
		\frac{L_3}{J_e} &  0 \\
		0               &  0 \\
		0               &  0                                
	}
\end{equation}

By writing out the system equations, we get 
\begin{subequations}\label{eq:model_almost_linearized}
	\begin{align}
		\ddot{\tilde p} &= \frac{L_1}{J_p} \tilde V_{d} \label{eq:model_almost_linearized_pitch}\\
		\ddot{\tilde e} &= \frac{L_3}{J_e} \tilde V_s \label{eq:model_almost_linearized_elev}\\
		\ddot{\tilde \lambda} &= -\frac{L_2 L_4}{L_3 J_\lambda} \tilde p \label{eq:model_almost_linearized_lambda}
	\end{align}
\end{subequations}

We assume the the moments of inertia for the helicopter are constant, meaning $J_p$, $J_e$ and $J_\lambda$ are given with these values:
\begin{subequations}
    \label{eq:linearized_momentum}
    \begin{align}
        J_p &= 2 m_p l_p ^2 \\
        J_e &= m_c l_c ^2 + 2 m_p l_h ^2 \\
        J_\lambda &= m_c l_c ^2 + 2 m_p (l_h ^2 + l_p ^2)
    \end{align}
\end{subequations}
These values come from the lab assignment. Combining (\ref{eq:model_almost_linearized}) and (\ref{eq:linearized_momentum}) gives us the system on the following form. 

\begin{subequations}\label{eq:model_linearized}
	\begin{align}
		\ddot{\tilde p} &= K_1 \tilde V_{d} \label{eq:model_linearized_pitch}\\
		\ddot{\tilde e} &= K_2 \tilde V_s \label{eq:model_linearized_elev}\\
		\ddot{\tilde \lambda} &= K_3 \tilde p \label{eq:model_linearized_lambda}
	\end{align}
\end{subequations}

with the constants $K_1$, $K_2$ and $K_3$ given by
\begin{subequations}
    \begin{align}
        K_1 &= \frac{K_f}{2 m_p l_p} \\
        K_2 &= \frac{K_f l_h}{m_c l_c ^2 + 2 m_p l_h ^2} \\
        K_3 &= -\frac{F_{g, c} l_c - (F_{g, f} + F_{g, b}) l_h}{L_3 m_c l_c ^2 + 2 m_p (l_h ^2 + l_p ^2)}
    \end{align}
\end{subequations}

%%%%%%%%%%%%%%%%%%%%%%%%%%%%%%%%%%%%%%%%%%%%%%%%%%%%%%%%%%%%%%%%%%%%%%%%%%%%%%%%%%%%%%%%%%%%

%%%%                                PART 2 PROBLEM 1

%%%%%%%%%%%%%%%%%%%%%%%%%%%%%%%%%%%%%%%%%%%%%%%%%%%%%%%%%%%%%%%%%%%%%%%%%%%%%%%%%%%%%%%%%%%%

\subsection{Part 2: Monovariable Control}
\subsubsection{Problem 1}

The PD-controller is given by
\begin{equation} \label{eq:part2_prob1_PD}
    \tilde{V}_d = K_{pp} (\tilde{p}_c - \tilde{p}) - K_{pd} \dot{\tilde{p}}
\end{equation}

where $K_{pp}$ and $K_{pd}$ are controller gains, and $\tilde{p}_c$ is the pitch angle reference. By substituting eq. (\ref{eq:part2_prob1_PD}) into the linearized model of elevation given by eq. (\ref{eq:model_linearized_pitch}), we get

\begin{equation} \label{eq:part2_prob1_pitch_with_pd}
    \ddot{\tilde p} = K_1 K_{pp} (\tilde{p}_c - \tilde{p}) - K_{pd} \dot{\tilde{p}}
\end{equation}

By applying the Laplace transform to eq. (\ref{eq:part2_prob1_pitch_with_pd}), we obtain the transfer function given by
\begin{equation}
    \frac{\tilde p}{\tilde p_c} = \frac{1}{1 + s \frac{K_{pd}}{K_{pp}} + \frac{s^2}{K_1 K_{pp}}}
\end{equation}

In order to choose suitable values for $K_{pp}$ and $K_{pd}$, we write the denominator on the form $1 + s \left(\frac{2 \zeta}{\omega_0} \right) + s^2 \left( \frac{1}{\omega_0^2} \right)$. We want our system to be critically damped, which can be achieved when $\zeta = 1$. Using this, we can express $K_{pp}$ and $K_{pd}$ as functions of $\omega_0$ in the following way:

\begin{subequations} \label{eq:part2_prob1_K}
    \begin{align}
        K_{pp} &= \frac{\omega_0^2}{K_1} \label{eq:part2_prob2_pd_1} \\
        K_{pd} &= \frac{2 \omega_0}{K_1} \label{eq:part2_prob2_pd_2}
    \end{align}
\end{subequations}

%%%%%%%%%%%%%%%%%%%%%%%%%%%%%%%%%%%%%%%%%%%%%%%%%%%%%%%%%%%%%%%%%%%%%%%%%%%%%%%%%%%%%%%%%%%%

%%%%                                PART 2 PROBLEM 2

%%%%%%%%%%%%%%%%%%%%%%%%%%%%%%%%%%%%%%%%%%%%%%%%%%%%%%%%%%%%%%%%%%%%%%%%%%%%%%%%%%%%%%%%%%%%

\subsubsection{Problem 2}
We want to control the travel with a P controller. Travel is controlled by changing the pitch. This make the helicopter rotate around the travel joint. The controller model is given by:
\begin{equation}
    \tilde{p}_c = K_{rp} (\dot{\tilde{\lambda}}_c - \dot{\tilde{\lambda}})
\end{equation}

where $K_{rp} < 0$. Inserting this into the linearized equation of motion for travel, (\ref{eq:model_linearized_lambda}) results in

\begin{equation}
    \ddot{\tilde \lambda} = K_3 K_{rp} (\dot{\tilde{\lambda}}_c - \dot{\tilde{\lambda}})
\end{equation}

assuming that the pitch is perfectly controlled, i.e. $\tilde{p} = \tilde{p}_c$. By applying the Laplace transform, the transfer function from travel rate reference $\dot{\tilde{\lambda}}_c$ to travel rate $\dot{\tilde{\lambda}}$ is given by
\begin{equation}
    \frac{\dot{\tilde \lambda}}{\dot{\tilde{\lambda}}_c} = \frac{K_3 K_{rp}}{s + K_3 K_{rp}}
\end{equation}


%%%%%%%%%%%%%%%%%%%%%%%%%%%%%%%%%%%%%%%%%%%%%%%%%%%%%%%%%%%%%%%%%%%%%%%%%%%%%%%%%%%%%%%%%%%%

%%%%                                PART 3 PROBLEM 1

%%%%%%%%%%%%%%%%%%%%%%%%%%%%%%%%%%%%%%%%%%%%%%%%%%%%%%%%%%%%%%%%%%%%%%%%%%%%%%%%%%%%%%%%%%%%

\subsection{Part 3: Multivariable Control}

\subsubsection{Problem 1}

We want to put eq. (\ref{eq:model_linearized_pitch}) and (\ref{eq:model_linearized_elev}) in state-space form, 
\begin{equation} \label{eq:state_space}
    \dot{\bm{x}} = \bm{A}\bm{x} + \bm{B}\bm{u}
\end{equation}

Given $ \bm{x} = [\tilde{p}_c,\, \dot{\tilde{p}}_c,\, \dot{\tilde e}_c]^\top$ and $\bm{u} = [\tilde V_s,\, \tilde V_d]^\top$ we get

\begin{equation} \label{eq:part3_prob1_A}
    \bm{A} = 
	\begin{bmatrix}
		0 &  1 &  0 \\
		0 &  0 &  0 \\
		0 &  0 &  0 
	\end{bmatrix}
\end{equation}

and
\begin{equation} \label{eq:part3_prob1_B}
    \bm{B} = 
	\begin{bmatrix}
		0   &  0   \\
		0   &  K_1 \\
		K_2 &  0 
	\end{bmatrix}
\end{equation}



%%%%%%%%%%%%%%%%%%%%%%%%%%%%%%%%%%%%%%%%%%%%%%%%%%%%%%%%%%%%%%%%%%%%%%%%%%%%%%%%%%%%%%%%%%%%

%%%%                                PART 3 PROBLEM 2

%%%%%%%%%%%%%%%%%%%%%%%%%%%%%%%%%%%%%%%%%%%%%%%%%%%%%%%%%%%%%%%%%%%%%%%%%%%%%%%%%%%%%%%%%%%%

\subsubsection{Problem 2}

We first want to check the controllability of the system in eq. (\ref{eq:state_space}) with $\bm{A}$ and $\bm{B}$ given by (\ref{eq:part3_prob1_A}) and (\ref{eq:part3_prob1_B}). In order to check if the system is controllable, we need to compute the controllability matrix $\mathcal{C} = [\bm{B} \ \bm{A}\bm{B} \ ...\ \bm{A}^{n-1}\bm{B}]$. In this case, to obtain the correct dimensions $n \times qn$, where $n$ is the number of rows in $\bm{A}$ and $q$ is the number of columns in $\bm{B}$, $\mathcal{C}$ is found to be

\begin{equation}
    \mathcal{C} = 
    \begin{bmatrix}
        0       & 0     & 0     & K_1   & 0     & 0 \\
        0       & K_1   & 0     & 0     & 0     & 0 \\
        K_2     & 0     & 0     & 0     & 0     & 0 \\
    \end{bmatrix}
\end{equation}

In order to be controllable, $\mathcal{C}$ has to have rank equal to $n$. As our controllability matrix $\mathcal{C}$ has $\mathrm{rank}(\mathcal{C}) = 3 = n$, the system is controllable.\medskip

We use the controller
\begin{equation} \label{eq:part3_prob2_state_feedback}
    \bm{u} = -\bm{K}\bm{x} + \bm{P}\bm{r}
\end{equation}

where $\bm{K}$ is the linear quadratic regulator. $\bm{K}$ is defined so that $\bm{u} = - \bm{K}\bm{x}$ optimizes the cost function
\begin{equation} \label{eq:cost_function}
    J = \int_0^\infty (\bm{x}^\top(t) \bm{Q} \bm{x}(t) +\bm{u}^\top(t) \bm{R} \bm{u} (t) ) \,dt 
\end{equation}

The cost function calculates a cost for the system. The cost increases when the error for the states increases, and when input is applied. When tuning the regulator we aim to minimize the system cost, thus having a small error using little input. \medskip

$\bm{Q}$ and $\bm{R}$ are diagonal weighting matrices. Along the diagonal are values for each of the states or inputs. By increasing the values in $\bm{Q}$, the cost for the error increases. This will force the regulator to minimize the error. Increasing the values in $\bm{R}$ increases the cost for using input, forcing the regulator to be more careful on input usage. So by tuning individual values in these matrices, we can control which aspects of the system the regulator prioritizes.\medskip

After deciding what the $\bm{Q}$ and $\bm{R}$ matrices should be, we use the matlab command \texttt{lqr(A, B, Q, R)} which returns the corresponding $\bm{K}$ matrix.
\medskip

Now that we have the $\bm{K}$ matrix, we can substitute $\bm{u}$ from eq. (\ref{eq:part3_prob2_state_feedback}) into eq. (\ref{eq:state_space}), giving us the following system:

\begin{equation}
    \dot{\bm{x}} = (\bm{A} - \bm{BK})\bm{x} + \bm{B}\bm{P}\bm{r}
\end{equation}

We want to track the reference $\bm{r} = [\tilde{p}_c, \, \dot{\tilde{e}}_c] ^\top$. That is, we want our output $\bm{y}$, given as

\begin{equation}\label{eq:y_p3_p2}
    \bm{y} = \bm{C} \bm{x} + \bm{D} \bm{u}
\end{equation}

to follow the reference $\bm{r}$. The output is $\bm{y} = [\tilde{p}, \, \dot{\tilde{e}}]^T$, and we find the matrices $\bm{C}$ and $\bm{D}$ to be

\begin{align}
    \bm{C} &= 
	\begin{bmatrix}
		1   &  0 &  0 \\
		0   &  0 &  1 \\
	\end{bmatrix}\\
	\bm{D} &= 0
\end{align}

We want to choose $\bm{P}$ such that $\lim_{t\to\infty} \tilde{p} = \tilde{p}_c$ and  $\lim_{t\to\infty} \dot{\tilde{e}} = \dot{\tilde{e}}_c$, with fixed values for $\bm{\tilde{p}_c}$ and $\bm{\dot{\tilde{e}}_c}$. This implies $\lim_{t\to\infty} \dot{\tilde{p}} = 0$, thus $\boldsymbol{\dot{\mathrm{x}}} = 0$. \medskip

%With $\bm{r} = [\tilde{p}_c, \, \dot{\tilde{e}}_c]$

Setting $\boldsymbol{\dot{\mathrm{x}}} = 0$ in eq. (\ref{eq:state_space}) gives the following equation
\begin{equation}
    \bm{0} = (\bm{A} - \bm{BK})\bm{x}_\infty + \bm{B}\bm{P}\bm{r}
\end{equation}

where $\bm{x}_\infty$ is the value of $\bm{x}$ when $\bm{\dot{x}} = 0$. Solving for $\bm{x}_\infty$ and combine with eq. (\ref{eq:y_p3_p2}) allows us to find an equation for P.

\begin{equation}
    \bm{y}_\infty = \bm{r} = \bm{C}\bm{x}_\infty = \bm{C}(\bm{B}\bm{K}-\bm{A})^{-1}\bm{B}\bm{P}\bm{r}
\end{equation}

We solve for $\bm{P}$:

\begin{equation}
    \label{eq:Pmatrix}
    \bm{P} = (\bm{C} ( \bm{B}\bm{K} - \bm{A}) ^{-1} \bm{B})^{-1}
\end{equation}

%%%%%%%%%%%%%%%%%%%%%%%%%%%%%%%%%%%%%%%%%%%%%%%%%%%%%%%%%%%%%%%%%%%%%%%%%%%%%%%%%%%%%%%%%%%%

%%%%                                PART 3 PROBLEM 3

%%%%%%%%%%%%%%%%%%%%%%%%%%%%%%%%%%%%%%%%%%%%%%%%%%%%%%%%%%%%%%%%%%%%%%%%%%%%%%%%%%%%%%%%%%%%

\subsubsection{Problem 3}
We now want to add integral effect to our system.
Given $\bm{x}_i = [\tilde{p}_c,\, \dot{\tilde{p}}_c,\, \dot{\tilde e}_c,\, \dot{\gamma},\, \dot{\zeta}]^\top$ where $\gamma$ and $\zeta$ are given by

\begin{align}
    \dot{\gamma} &= \tilde p - {\tilde p}_c \\
    \dot{\zeta} &= \dot{\tilde e} - \dot{{\tilde e}}_c
\end{align}


and subscript $i$ denote integral effect, we get a new state space equation

\begin{equation} \label{eq:part3_prob3_5beer}
    \dot{\bm{x}}_i = \bm{A}_i \bm{x}_i + \bm{B}_i \bm{u}_i + \bm{F}_i \bm{r}_i
\end{equation}

with 

\begin{equation} \label{eq:A_pi}
    \bm{A}_i = 
	\begin{bmatrix}
		0 &  1 &  0 &  0 &  0 \\
		0 &  0 &  0 &  0 &  0 \\
		0 &  0 &  0 &  0 &  0 \\
		1 &  0 &  0 &  0 &  0 \\
		0 &  0 &  1 &  0 &  0
	\end{bmatrix}
\end{equation}

\begin{equation} \label{eq:B_pi}
    \bm{B}_i = 
	\begin{bmatrix}
		0   &  0 \\
		0   &  K_1 \\
		K_2 &  0 \\
		0   &  0 \\
		0   &  0
	\end{bmatrix}
\end{equation}


\begin{equation}
    \bm{F}_i = 
	\begin{bmatrix}
		0   &  0 \\
		0   &  0 \\
		0   &  0 \\
		1   &  0 \\
		0   &  1
	\end{bmatrix}
\end{equation} 

With this new model, we also need to calculate new matrices for the controller. This means generating a new $\bm{K}$ and $\bm{P}$ matrix. We get $\bm{K}$ from the LQR matlab command. $\bm{P}$ is calculated with \cref{eq:Pmatrix}. Because our system has a lot of zeroes, calculating $\bm{P}$ with out system matrices doesn't work. We get a singular matrix, which cannot be inverted.\medskip

During the Linear systems lecture, we learned to ignore the integral states when calculating $\bm{P}$ to avoid this problem, as they are not physical states in our system but aid our mathematical model. We therefor sliced the 5x5 matrices to 3x3 matrices, and used this to calculate the controller matrices.

\subsection{Part 4: State Estimation}

%%%%%%%%%%%%%%%%%%%%%%%%%%%%%%%%%%%%%%%%%%%%%%%%%%%%%%%%%%%%%%%%%%%%%%%%%%%%%%%%%%%%%%%%%%%%

%%%%                                PART 4 PROBLEM 1

%%%%%%%%%%%%%%%%%%%%%%%%%%%%%%%%%%%%%%%%%%%%%%%%%%%%%%%%%%%%%%%%%%%%%%%%%%%%%%%%%%%%%%%%%%%%

\subsubsection{Problem 1}

We start by deriving a state space model on the form 
\begin{align}
    \bm{\dot{x}} &= \bm{A}\bm{x} + \bm{B}\bm{u}\\
    \bm{y} &= \bm{C}\bm{x}
\end{align}

based on eq. (\ref{eq:model_linearized}), the linearized model of the system, where $\bm{x} = [\tilde{p}, \, \dot{\tilde{p}}, \, \tilde{e}, \, \dot{\tilde{e}}, \, \tilde{\lambda}, \, \dot{\tilde{\lambda}}]^\top$.

\begin{equation} \label{eq:part4_prob1_A}
    \bm{A} = 
	\begin{bmatrix}
		0   & 1   & 0   & 0   & 0   & 0 \\
		0   & 0   & 0   & 0   & 0   & 0 \\
		0   & 0   & 0   & 1   & 0   & 0 \\
		0   & 0   & 0   & 0   & 0   & 0 \\
		0   & 0   & 0   & 0   & 0   & 1 \\
		K_3 & 0   & 0   & 0   & 0   & 0 \\
	\end{bmatrix}
\end{equation} 


\begin{equation} \label{eq:part4_prob1_B}
    \bm{B} = 
	\begin{bmatrix}
		0   & 0   \\
		0   & K_1 \\
		0   & 0   \\
		K_2 & 0   \\
		0   & 0   \\
		0   & 0   \\
	\end{bmatrix}
\end{equation} 

\begin{equation} \label{eq:part4_prob1_C}
    \bm{C} = 
	\begin{bmatrix}
		1   & 0   & 0   & 0   & 0   & 0 \\
		0   & 0   & 1   & 0   & 0   & 0 \\
		0   & 0   & 0   & 0   & 1   & 0 \\
	\end{bmatrix}
\end{equation} 


%%%%%%%%%%%%%%%%%%%%%%%%%%%%%%%%%%%%%%%%%%%%%%%%%%%%%%%%%%%%%%%%%%%%%%%%%%%%%%%%%%%%%%%%%%%%

%%%%                                PART 4 PROBLEM 2

%%%%%%%%%%%%%%%%%%%%%%%%%%%%%%%%%%%%%%%%%%%%%%%%%%%%%%%%%%%%%%%%%%%%%%%%%%%%%%%%%%%%%%%%%%%%
\subsubsection{Problem 2}

We begin by examining the observability of the system. We look at the observability matrix $\mathcal{O}$, defined as the $nq \times n$ matrix 
\begin{equation}
\mathcal{O} = 
    \begin{bmatrix}
        \bm{C} \\
        \bm{C} \bm{A} \\
        \vdots \\
        \ \bm{C} \bm{A}^{n-1} \ 
    \end{bmatrix}
\end{equation}

where $n$ is the number of columns in $\bm{A}$ and $q$ is the number of rows in $\bm{C}$. With $\bm{A}$ and $\bm{C}$ from eqs. (\ref{eq:part4_prob1_A}) and (\ref{eq:part4_prob1_C}), $\mathcal{O}$ becomes a $18 \times 6$ matrix. We used the matlab command \texttt{obsv()} (see appendix \ref{sec:matlab_observability}) to obtain the following:

\begin{equation} \label{eq:part4_prob2_O}
    \setstackgap{L}{1.1\baselineskip}
    \fixTABwidth{T}
	\mathcal{O} = \bracketMatrixstack{
		1   & 0   & 0   & 0   & 0   & 0 \\ %1
		0   & 0   & 1   & 0   & 0   & 0 \\ %2
		0   & 0   & 0   & 0   & 1   & 0 \\ %3
		0   & 1   & 0   & 0   & 0   & 0 \\ %4
		0   & 0   & 0   & 1   & 0   & 0 \\ %5
		0   & 0   & 0   & 0   & 0   & 1 \\ %6
		0   & 0   & 0   & 0   & 0   & 0 \\ %7
		0   & 0   & 0   & 0   & 0   & 0 \\ %8
		K_3 & 0   & 0   & 0   & 0   & 0 \\ %9
		0   & 0   & 0   & 0   & 0   & 0 \\ %10
		0   & 0   & 0   & 0   & 0   & 0 \\ %11
		0   & K_3 & 0   & 0   & 0   & 0 \\ %12
		0   & 0   & 0   & 0   & 0   & 0 \\ %13
		0   & 0   & 0   & 0   & 0   & 0 \\ %14
		0   & 0   & 0   & 0   & 0   & 0 \\ %15
		0   & 0   & 0   & 0   & 0   & 0 \\ %16
		0   & 0   & 0   & 0   & 0   & 0 \\ %17
		0   & 0   & 0   & 0   & 0   & 0    %18
	}
\end{equation} 

We see from (\ref{eq:part4_prob2_O}) that $\mathrm{rank}(\mathcal{O})=6$, and the system is thus observable. We can now create a linear system on the form

\begin{equation}
    \bm{\dot{\hat{x}}} = \bm{A}\bm{\hat{x}} + \bm{B}\bm{u} + \bm{L} (\bm{y} - \bm{C}\bm{\hat{x}})
\end{equation}

where $\bm{\hat{x}}$ is the observer and $\bm{L}$ is the observer gain matrix. We use pole placement to place the poles, done with the matlab command \texttt{poles(A', C', poles)'}, where \texttt{poles} is a list of desired poles and \texttt{'} represent the matrix transpose.


%%%%%%%%%%%%%%%%%%%%%%%%%%%%%%%%%%%%%%%%%%%%%%%%%%%%%%%%%%%%%%%%%%%%%%%%%%%%%%%%%%%%%%%%%%%%

%%%%                                PART 4 PROBLEM 3

%%%%%%%%%%%%%%%%%%%%%%%%%%%%%%%%%%%%%%%%%%%%%%%%%%%%%%%%%%%%%%%%%%%%%%%%%%%%%%%%%%%%%%%%%%%%
\subsubsection{Problem 3} \label{sec:P4p3}

We now wish to examine the observability of the system for $\bm{y} = [\tilde{e}, \, \tilde{\lambda}]^\top$ and  $\bm{y}_2 = [\tilde{p}, \, \tilde{e}]^\top$. This gives us new $\bm{C}$ matrices 
\begin{equation} \label{eq:part4_prob3_C}
    \bm{C} = 
	\begin{bmatrix}
		0   & 0   & 1   & 0   & 0   & 0 \\
		0   & 0   & 0   & 0   & 1   & 0
	\end{bmatrix}
\end{equation} 
and
\begin{equation} \label{eq:part4_prob3_C2}
    \bm{C}_2 = 
	\begin{bmatrix}
		1   & 0   & 0   & 0   & 0   & 0 \\
		0   & 0   & 1   & 0   & 0   & 0
	\end{bmatrix}
\end{equation} 

We can use the same matlab command as in part 4, problem 2, to examine the observability of the systems. The observability matrices are

\begin{equation} \label{eq:part4_prob3_O}
	\setstackgap{L}{1.1\baselineskip}
    \fixTABwidth{T}
	\mathcal{O} = \bracketMatrixstack{
		0   & 0   & 1   & 0   & 0   & 0 \\ %1
		0   & 0   & 0   & 0   & 1   & 0 \\ %2
		0   & 0   & 0   & 1   & 0   & 0 \\ %3
		0   & 0   & 0   & 0   & 0   & 1 \\ %4
		0   & 0   & 0   & 0   & 0   & 0 \\ %5
		K_3 & 0   & 0   & 0   & 0   & 0 \\ %6
		0   & 0   & 0   & 0   & 0   & 0 \\ %7
		0   & K_3 & 0   & 0   & 0   & 0 \\ %8
		0   & 0   & 0   & 0   & 0   & 0 \\ %9
		0   & 0   & 0   & 0   & 0   & 0 \\ %10
		0   & 0   & 0   & 0   & 0   & 0 \\ %11
		0   & 0   & 0   & 0   & 0   & 0    %12
	}
\end{equation}
and
\begin{equation} \label{eq:part4_prob3_O2}
	\mathcal{O}_2 = \begin{bmatrix}
		1   & 0   & 0   & 0   & 0   & 0 \\ %1
		0   & 0   & 1   & 0   & 0   & 0 \\ %2
		0   & 1   & 0   & 0   & 0   & 0 \\ %3
		0   & 0   & 0   & 1   & 0   & 0 \\ %4
		0   & 0   & 0   & 0   & 0   & 0 \\ %5
		0   & 0   & 0   & 0   & 0   & 0 \\ %6
		0   & 0   & 0   & 0   & 0   & 0 \\ %7
		0   & 0   & 0   & 0   & 0   & 0 \\ %8
		0   & 0   & 0   & 0   & 0   & 0 \\ %9
		0   & 0   & 0   & 0   & 0   & 0 \\ %10
		0   & 0   & 0   & 0   & 0   & 0 \\ %11
		0   & 0   & 0   & 0   & 0   & 0    %12
	\end{bmatrix}
\end{equation} 



We see from (\ref{eq:part4_prob3_O}) that $\mathrm{rank}(\mathcal{O}) = 6$ and from (\ref{eq:part4_prob3_O2}) that $\mathrm{rank}(\mathcal{O}_2) = 4$, that is, the system is not observable when only the pitch and elevation is measured. However if we measure elevation and travel, the system is observable. Since $\bm{y}$ and $\bm{C}$ have new dimensions, we need a new $\bm{L}$. This can be obtained by the same \texttt{place()} command as before.